%Конечный автомат
\section{Конечно-автоматная модель}

 Так как система тел при движении во времени может принимать несколько состояний, то мы можем рассмотреть данную систему в качестве  конечного автомата с тремя возможными состояниями, описывающий динамическое состояние системы. Всего возможно три состояния системы:
\begin{enumerate}
	\item тело взаимодействует со стенкой;
	\item тело не взаимодействует со стенкой (свободное движение);
	\item нефизический случай.
\end{enumerate}

При наличии взаимодействия со стенкой в уравнении появляется пара сил: сила упругости первой пружины $F$ и, согласно третьему закону Ньютона действует обратная сила со стороны стенки.Ппри отсутствии взаимодействия, очевидно, данная пара сил не существует, поэтому в состоянии оторванном от стенки считается, что первая пружина отсутствует. Взаимодействие со стенкой происходит, когда значение координаты массы $m_1$ меньше или равняется длине пружины  $l$ в ненапряженном состоянии.

 Нефизическим случаем является такое состояние системы, при котором значение координаты массы $m_{i+1}$ становятся меньше значения координаты массы $m_i$,изначально находящейся ближе к стенке. Также нефизическим случаем является, когда значение координаты какой-либо массы становиться меньше 0 (масса "прошла" сквозь стенку). Граф, описывающий конечный автомат, представлен на рисунке 
 \ref{fig:avtomat}. Программная реализация автомата представлена на блок-схеме \ref{alg:konavt}.



\begin{lstlisting}[numbers=none,label=lst:konavt,language=Golang, caption=Программная реализация конечного автомата]
	//UpdateState - checks the conditions of state machine
func (st *StateMachine) UpdateState() error {
	cond := ConditionFromVec(*st.CurCond)
	if cond.X[0] <= -st.Length || cond.X[1] <= -2*st.Length || cond.X[2] <= -3*st.Length || cond.X[1] <= cond.X[0]-st.Length || cond.X[2] <= cond.X[1]-st.Length {
		st.CurState = NonPhis
	}
	switch st.CurState {
	case Started:
		if !st.Mute {
			fmt.Printf("HitWall: %v sec\n", st.CurTime)
		}
		st.CurState = HitWall
	case HitWall:
		if cond.X[0] > 0 {
			if !st.Mute {
				fmt.Printf("BouncedBack: %v sec\n", st.CurTime)
			}
			st.CurState = BouncedBack
		}
	case BouncedBack:
		if cond.X[0] <= 0. {
			if !st.Mute {
				fmt.Printf("HitWall: %v sec\n", st.CurTime)
			}
			st.CurState = HitWall
		}
	case NonPhis:
		return errors.New("NonPhis conditions reached")
	}
	if st.CurTime >= st.FinTime {
		return errors.New("Final time reached, simulation stopped")
	}
	return nil
}
\end{lstlisting}

%\begin{figure}[b!]
%    \centering
%    \resizebox{\textwidth}{0.4\textwidth}{%
%    \begin{tikzpicture}[shorten >=1pt,node distance=5cm,on grid, auto, scale=0.2] 
%        \node[state,initial] (q_0) {
%            \begin{tabular}{c}
%                Свободное \\ движение тела
%            \end{tabular}
%        };
%        \node[state] (q_1) [right=of q_0] {
%            \begin{tabular}{c}
%                Тело \\ взаимодействует \\со стенкой
%            \end{tabular}
%        };
%        \node[state] (q_2) [below=of q_0] {
%            \begin{tabular}{c}
%                Нефизический\\ случай
%            \end{tabular}
%        };
%        \path[->, bend right = 30] (q_0) edge node [below] {$x_1 \leq l$} (q_1);
%        \path[->, bend right = 30] (q_1) edge node [above] {$x_1 > l$} (q_0);
%        \path[->, bend right = 30] (q_0) edge node [left] {
%            \begin{tabular}{c}
%                $x_2\leq x_1$, \\
%                $x_3\leq x_2$, \\
%                $x_1\leq 0$  \\
%            \end{tabular}
%        } (q_2);
%        \path[->, bend left = 30] (q_1) edge node [below right] {
%           \begin{tabular}{c}
%                $x_2\leq x_1$ \\
%                $x_3\leq x_2$ \\
%                $x_1\leq 0$  \\
%            \end{tabular}
%        } (q_2);
%    \end{tikzpicture}
%    }
%    \caption{Конечный автомат, описывающий динамическое состояние системы трех масс, взаимодействующих со стенкой.}
%    \label{fig:avtomat}
%\end{figure}
%
