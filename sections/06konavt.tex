%Конечный автомат
\section{Конечно-автоматная модель}

 Так как система тел при движении во времени может принимать несколько состояний, то мы можем рассмотреть данную систему в качестве  конечного автомата с тремя возможными состояниями, описывающий динамическое состояние системы. Всего возможно три состояния системы:
\begin{enumerate}
	\item тело взаимодействует со стенкой;
	\item тело не взаимодействует со стенкой (свободное движение);
	\item нефизический случай.
\end{enumerate}

При наличии взаимодействия со стенкой в уравнении появляется пара сил: сила упругости первой пружины $F$ и, согласно третьему закону Ньютона действует обратная сила со стороны стенки.Ппри отсутствии взаимодействия, очевидно, данная пара сил не существует, поэтому в состоянии оторванном от стенки считается, что первая пружина отсутствует. Взаимодействие со стенкой происходит, когда значение координаты массы $m_1$ меньше или равняется длине пружины  $l$ в ненапряженном состоянии.

 Нефизическим случаем является такое состояние системы, при котором значение координаты массы $m_{i+1}$ становятся меньше значения координаты массы $m_i$,изначально находящейся ближе к стенке. Также нефизическим случаем является, когда значение координаты какой-либо массы становиться меньше 0 (масса "прошла" сквозь стенку). Граф, описывающий конечный автомат, представлен на рисунке 
 \ref{fig:avtomat}. Программная реализация автомата представлена на блок-схеме \ref{fig:konavtbs}.


\newpage
\begin{figure}
    \centering
    \scalebox{0.5}{\begin{tikzpicture}[start chain=going below,node distance=6mm, column sep=2cm]
    \node [cloud] (start) {Начало};
    \node [block, join] (init1) {Задание начальных значений};
    \node [block, join] (init2) {State := STARTED};

    \node [cyclebegin, join] (cycbeg) {time = 0; time <= tfin};
    \node [subroutine, join] (chck) {Проверка\\ состояния};
    \node [block, join] (stat) {State = Current State};
    \node [block, join] (step) {Вычисление значений\\ скоростей и координат\\ на следующем шаге};
    \node [io, join] (write ) {Вывод полученных\\ значений в файл};
    \node [cycleend, join] (cycend) {time+=time step};
    
    \node [cloud, join] (finish) {Конец};
\end{tikzpicture}}
    \caption{Блок-схема, описывающая программную реализацию конечного автомата}
    \label{fig:konavtbs}
\end{figure}

\begin{figure}
    \centering
    \scalebox{0.6}{\begin{tikzpicture}[shorten >=1pt,node distance=5cm,on grid, auto, scale=0.2] 
        \node[state,initial] (q_0) {
            \begin{tabular}{c}
                Свободное \\ движение тела
            \end{tabular}
        };
        \node[state] (q_1) [right=of q_0] {
            \begin{tabular}{c}
                Тело \\ взаимодействует \\со стенкой
            \end{tabular}
        };
        \node[state] (q_2) [below=of q_0] {
            \begin{tabular}{c}
                Нефизический\\ случай
            \end{tabular}
        };
        \path[->, bend right = 30] (q_0) edge node [below] {$x_1 \leq l$} (q_1);
        \path[->, bend right = 30] (q_1) edge node [above] {$x_1 > l$} (q_0);
        \path[->, bend right = 30] (q_0) edge node [left] {
            \begin{tabular}{c}
                $x_2\leq x_1$, \\
                $x_3\leq x_2$, \\
                $x_1\leq 0$  \\
            \end{tabular}
        } (q_2);
        \path[->, bend left = 30] (q_1) edge node [below right] {
           \begin{tabular}{c}
                $x_2\leq x_1$ \\
                $x_3\leq x_2$ \\
                $x_1\leq 0$  \\
            \end{tabular}
        } (q_2);
    \end{tikzpicture}}
    \caption{Конечный автомат, описывающий динамическое состояние системы трех масс, взаимодействующих со стенкой}
    \label{fig:avtomat}
\end{figure}

