
%Математическая модель
\chapter{МАТЕМАТИЧЕСКАЯ ПОСТАНОВКА ЗАДАЧИ}
\section{Математическая модель}

Система, показанная на рисунке \ref{fig:sketch} может быть описана следующей системой уравнений движения:
\begin{align}
	m_1 \ddot{x}_1 &= k_1 (l - x_1) - k_2 \left[ l - (x_2 - x_1) \right], \\
	m_2 \ddot{x}_2 &= k_2 \left[ l - (x_2 - x_1) \right] - k_3 \left[ l - (x_3 - x_2) \right], \\
	m_3 \ddot{x}_3 &= k_3 \left[ l - (x_3 - x_2) \right].
\end{align}
Сдвинув координаты, получим:
\begin{align}
	x_1 &= \tilde{x}_1 + l, \\
	x_2 &= \tilde{x}_2 + 2l, \\
	x_3 &= \tilde{x}_3 + 3l,
\end{align}
и, подставив полученные значения в уравнения движения, получим:
\begin{align}
	m_1 \ddot{\tilde{x}}_1 &= -k_1 \tilde{x}_1 + k_2 (\tilde{x}_2 - \tilde{x}_1), \\
	m_2 \ddot{\tilde{x}}_2 &= -k_2 (\tilde{x}_2 - \tilde{x}_1) + k_3 (\tilde{x}_3 - \tilde{x}_2), \\
	m_3 \ddot{\tilde{x}}_3 &= -k_3 (\tilde{x}_3 - \tilde{x}_2).
\end{align}
Разделив уравнения на соответствующие массы получим новые переменные: $\omega_i^2 = \frac{k_i}{m_i}$ and $\omega_{ij}^2 = \frac{k_i}{m_j}$. В итоге система уравнений имеет вид:
\begin{align}
	\ddot{\tilde{x}}_1 &= -\omega_1^2 \tilde{x}_1 + \omega_{21}^2 (\tilde{x}_2 - \tilde{x}_1), \\
	\ddot{\tilde{x}}_2 &= -\omega_2^2 (\tilde{x}_2 - \tilde{x}_1) + \omega_{32}^2 (\tilde{x}_3 - \tilde{x}_2), \\
	\ddot{\tilde{x}}_3 &= -\omega_3^2 (\tilde{x}_3 - \tilde{x}_2).
\end{align}
Для составления системы ОДУ введем вектор $q$:
\begin{equation}\label{eq:vector}
	\bm{q} = [q_1, q_2, q_3, q_4, q_5, q_6]^T = [\tilde{x}_1, \tilde{x}_2, \tilde{x}_3, \dot{\tilde{x}}_4, \dot{\tilde{x}}_5, \dot{\tilde{x}}_6]^T.
\end{equation}
Таким образом система ОДУ $\frac{d}{dt}q$ имеет вид:
\begin{equation}
	\frac{d}{dt}
	\begin{bmatrix} q_1 \\ q_2 \\ q_3 \\ q_4 \\ q_5 \\ q_6 \end{bmatrix}
	=
	\begin{bmatrix}
		0 & 0 & 0 & 1 & 0 & 0 \\
		0 & 0 & 0 & 0 & 1 & 0 \\
		0 & 0 & 0 & 0 & 0 & 1 \\
		-\omega_1^2 - \omega_{21}^2 & \omega_{21}^2 & 0 & 0 & 0 & 0 \\
		\omega_2^2 & -\omega_{2}^2 - \omega_{32}^2 & \omega_{32}^2 & 0 & 0 & 0 \\
		0 & \omega_{3}^2 & -\omega_{3}^2 & 0 & 0 & 0 \\
	\end{bmatrix}
	\begin{bmatrix} q_1 \\ q_2 \\ q_3 \\ q_4 \\ q_5 \\ q_6 \end{bmatrix}
	\label{eq:syst}
\end{equation}

Таким образом нахождение значений вектора $q$, путем решения ОДУ, позволяет найти значения скоростей и координат масс $m_1$, $m_2$, $m_3$ в любой момент времени и понять какие процессы протекают в системе во время взаимодействия.  