\section{Параллелизация вычислений}

Задача параллелизации представляет собой программное ускорение расчета полей накопленных энергий после соударения системы трех тел со стенкой в зависимости
от конфигурации системы. Расчет энергий происходит путем моделирования взаимодействия тел, что представляет собой решение системы ОДУ для каждого момента
времени при помощи метода Рунге-Кутта 4 порядка \cite{BahvalJidkovKobel1987}. Для понимания какой из параметров вносит больший вклад в накопление энергии 
требуется провести большое количество симуляций, которые линейно влияют на время расчетов. Было решено использовать возможности языка Golang, для программного
ускорения времени расчетов и более полного использования ресурсов компьютера, а в дальнейшем и вычислительного кластера.

Для ускорения вычисления были реализованы функции считающие двумерные поля(листинг \ref{lst:2d}) и трехмерные облака точек(листинг \ref{lst:3d}).
Обе функции используют все ядра центрального процессора.



Параллелизация работы в случае функции представленной в листинге \ref{lst:3d} достигается за счет того, что на каждую горутину выделяется одна плоскость в трехмерном пространстве, 
которую необходимо расчитать, после окончания расчета по каналу отправляется строковая переменная содержащая в себе данные расчетов, полученные данной горутиной.

Параллелизация работы в случае функции представленной в листинге \ref{lst:2d} достигается, за счет разделения плоскости расчетов между всеми доступными ядрами процессора,
что позволяет параллельно производить несколько симуляций. Во время расчетов, аналогично трехмерному случаю формируется строковая переменная, которая отправляется по каналу.


\begin{lstlisting}[numbers=none, label=lst:bar,language=Golang, caption=Функция барьер][t]
    func barrier(nrouts int, ch chan string, fil *os.File) {
        for r := 0; r < nrouts; r++ {
            x := <-ch
            log.Println("goroutine ", r, "writed", len(x))
            fil.WriteString(x)
        }
    }
\end{lstlisting}

Полученные через канал строки обрабатываются в функции представленной в листинге \ref{lst:bar}. Данная функция принимает по каналу определенное количество переменных типа строка,
которое равно число запущенных горутин, тем самым реализует работу барьера и блокирует выполнение родительской горутины до завершения работы всех дочерних горутин.
После строковая переменная записывается в файл для дальнейшей визуализации.