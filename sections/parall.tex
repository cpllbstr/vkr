\section{Параллелизация вычислений}

Задача параллелизации представляет собой программное ускорение расчета полей накопленных энергий после соударения системы трех тел со стенкой в зависимости
от конфигурации системы. Расчет энергий происходит путем моделирования взаимодействия тел, что представляет собой решение системы ОДУ для каждого момента
времени при помощи метода Рунге-Кутта 4 порядка \cite{BahvalJidkovKobel1987}. Для понимания какой из параметров вносит больший вклад в накопление энергии 
требуется провести большое количество симуляций, которые линейно влияют на время расчетов. Было решено использовать возможности языка Golang, для программного
ускорения времени расчетов и более полного использования ресурсов компьютера, а в дальнейшем и вычислительного кластера.

Для ускорения вычисления были реализованы функции считающие двумерные поля(рисунок \ref{fig:2dalg}) и трехмерные облака точек(рисунок \ref{fig:3dalg}).
Обе функции используют все ядра центрального процессора.


Параллелизация работы в случае функции представленной в блок-схеме на рисунке \ref{fig:3dalg} достигается за счет того, что на каждую горутину выделяется одна плоскость в трехмерном пространстве, 
которую необходимо расчитать, после окончания расчета по каналу отправляется строковая переменная содержащая в себе данные расчетов, полученные данной горутиной.


\begin{figure}[!h]
    \centering
    \begin{minipage}[h]{0.49\linewidth}
        \centering
        \scalebox{0.45}{\begin{tikzpicture}[start chain=going below,node distance=6mm, column sep=2cm]
    \node [cloud] (start) {Начало};
    \node [block, join] (init1) {Определение количества\\доступных потоков};
    \node [block, join] (init2) {Вычисление шага сетки};
    \node [block, join] (init3) {Создание модели\\трех тел};
    \node [cyclebegin, join] (cycbeg) {Число горутин < NumCPU};
    \node [block, join] (chck) {start mass = 1 + n*step\\final mass := 1 + float64(n)*step};
    \node [subroutine, join] (gorout) {Variate(start mass, final mass)};
    \node [cycleend, join] (cycend) {Конец\\цикла};
    \node [subroutine, join] (bar) {Барьер};
    \node [cloud, join] (finish) {Конец};
\end{tikzpicture}
}
        \caption{Двумерная\\параллелизация}
        \label{fig:2dalg}
    \end{minipage}
    \begin{minipage}[h]{0.49\linewidth}
        \centering
        \scalebox{0.45}{\begin{tikzpicture}[start chain=going below,node distance=6mm, column sep=2cm]
    \node [cloud] (start) {Начало};
    \node [block, join] (init1) {Определение количества\\доступных потоков};
    \node [block, join] (init2) {Вычисление шага сетки};
    \node [cyclebegin, join] (stepm) {Z=0.1; Z<=total};
    \node [cyclebegin, join] (cycbeg) {Число горутин < NumCPU};
    \node [if, join] (cond) {Посчитаны все\\плоскости?};
    %НЕТ
    \node [block] (init3) {Создание модели\\трех тел};
    \node [block, join] (chck) {start mass = 1 + n*step\\final mass := 1 + float64(n)*step};
    \node [subroutine, join] (gorout) {Variate(0, total)};
    \node [cycleend, join] (stepme) {Z+=число запущенных горутин};
    \node [cycleend, join] (cycend) {Конец\\цикла};

    \node [subroutine, join] (bar) {Барьер};
    \node [cloud, join] (finish) {Конец};
    
    \node [io, left of =cycend, node distance = 6cm] (conn) {Отправление по каналу\\пустой строки};
    \path [line] (cond) -| node [u,near start] {Да} (conn);
    \path [line] (conn) |- (bar);
    \path [line] (cond) to node [r] {Нет} (init3);
\end{tikzpicture}
}
        \caption{Трехмерная\\параллелизация}
        \label{fig:3dalg}
    \end{minipage}
    %\scalebox{0.45}{\begin{tikzpicture}[start chain=going below,node distance=6mm, column sep=2cm]
    \node [cloud] (start) {Начало};
    \node [block, join] (init1) {Определение количества\\доступных потоков};
    \node [block, join] (init2) {Вычисление шага сетки};
    \node [block, join] (init3) {Создание модели\\трех тел};
    \node [cyclebegin, join] (cycbeg) {Число горутин < NumCPU};
    \node [block, join] (chck) {start mass = 1 + n*step\\final mass := 1 + float64(n)*step};
    \node [subroutine, join] (gorout) {Variate(start mass, final mass)};
    \node [cycleend, join] (cycend) {Конец\\цикла};
    \node [subroutine, join] (bar) {Барьер};
    \node [cloud, join] (finish) {Конец};
\end{tikzpicture}
}
    %\caption{Двумерная параллелизация}
    %\label{fig:2dalg}
\end{figure}

\begin{figure}[!h]
    \centering
    \scalebox{0.45}{\begin{tikzpicture}[start chain=going below,node distance=6mm, column sep=2cm]
    \node [cloud] (start) {Variate};
    \node [cyclebegin, join] (cycm1) {m1 = start mass; m1<=final mass}  ;
    \node [cyclebegin, join] (cycm2) {m2 = start mass; m2<=final mass};
    \node [subroutine, join] (sym)  {Симуляция соударения};
    \node [if, join] (nonph) {Достигнут нефизический случай?};

    \node [block] (energ) {Вычисление запасенной энергии};
    \node [cycleend, join] (cycm1e) {m1+=grid step}  ;
    \node [cycleend, join] (cycm2e) {m2+=grid step};
    \node [io,  join]  (out) {Отправление полученных\\значений по каналу};
    \node [cloud, join] (fin) {Конец};
    
    \node [coord,  left of = energ, node distance = 6cm] (conn) {}; 
    \path [line] (nonph) -| node [u,near start] {Да} (conn);
    \path [line] (conn) |- (cycm2e);
    \path [line] (nonph) to node [r] {Нет} (energ);
\end{tikzpicture}}
    \caption{Функция Variate}
    \label{fig:var2}
\end{figure}

Параллелизация работы в случае функции представленной в блок-схеме на рисунке \ref{fig:2dalg} достигается, за счет разделения плоскости расчетов между всеми доступными ядрами процессора,
что позволяет параллельно производить несколько симуляций. Во время расчетов, аналогично трехмерному случаю формируется строковая переменная, которая отправляется по каналу.

\begin{lstlisting}[label=lst:bar,language=Golang, caption=Функция барьер][t]
func barrier(nrouts int, ch chan string, fil *os.File) {
    for r := 0; r < nrouts; r++ {
        x := <-ch
        log.Println("goroutine ", r, "writed", len(x))
        fil.WriteString(x)
    }
}
\end{lstlisting}

Полученные через канал строки обрабатываются в функции представленной в листинге \ref{lst:bar}. Данная функция принимает по каналу определенное количество переменных типа строка,
которое равно число запущенных горутин, тем самым реализует работу барьера и блокирует выполнение родительской горутины до завершения работы всех дочерних горутин.
После строковая переменная записывается в файл для дальнейшей визуализации.