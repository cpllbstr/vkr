
%:Заключение
\chapter{ЗАКЛЮЧЕНИЕ}

В ходе работы были выполнены следующие задачи:
\begin{itemize}
	\item По теме "Силовые сети в гранулированных средах" был проведен обзор литературы и было обнаружено отсутствие подходящих исследований, описывающих проблематику проекта
	\item Определена актуальность проекта и определена цель исследования в соответствии с актуальными вопросами по теме проекта
	\item Составлена математическая модель неабсолютно упругого соударения системы трех тел со стенкой.
	\item Построен и реализован конечный автомат, описывающий состояния системы.
	\item Проведен анализ методов решения и выбор оптимального. 
	\begin{itemize}
		\item Реализовано аналитическое решение системы ОДУ
		\item Реализован метод Рунге-Кутта для решения ОДУ
		\item Произведен анализ эффективности методов и скорости расчетов.
	\end{itemize}
	\item Проведен анализ зависимости коэффициента эффективности диссипации от значений переменных системы.
	\begin{itemize}
		\item Были построены двумерные и трехмерные облака точек, показывающие зависимости коэффициента диссипации от параметров системы
		\item Определены принципы поведения системы трех тел в областях максимумов и минимумов энергий.  
	\end{itemize}
\end{itemize}

Данные, полученные в ходе работы над курсовым проектом, являются стартовой точкой для понимания процессов, происходящих в неабсолютно упругих телах при соударении. Данная математическая модель и понимание какие из параметров сильней всего влияют на состояние системы, позволят в дальнейшем найти такие значения параметров системы для которых накопленная внутренняя энергия будет максимальна, что позволит определить оптимальные параметры для полидисперсных демпфирующих систем. 
