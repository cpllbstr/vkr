%Аннотация
%%\newpage
%\addtocontents{toc}{\protect\setcounter{tocdepth}{-1}}
\chapter*{АННОТАЦИЯ}

Объектом исследования является механическая одномерная модель описывающая взаимодействие тела из полидисперсного материала с абсолютно твердой стенкой.
Работа посвящена исследованию процесса возникновения диссипации энергии в системе, моделирующей неабсолютно упругое соударение.
Диссипативная пара сил образуется при динамическом взаимодействии тела и стенки, в процессе преобразования кинетической энергии тела во внутреннюю энергию, а так же во время преобразования внутренней энергии тела в кинетическую.

Моделирование поведения полидисперсного материала производится с помощью
механической модели. Работа нацелена на изучение процессов диссипации в ходе упругого соударения, поэтому
исследуемая модель не учитывает потери энергии из-за трения и т.п. То есть в пределах
одного столкновения тел система является консервативной, соударения считаются
абсолютно упругими а тела абсолютно жесткими. В работе исследуется процесс
преобразования кинетической энергии тела во внутреннюю, при столкновении тела и стенки.

В ходе соударения упругого тела и стенки в теле на микроуровне возникают волны сжатия и растяжения, до тех пор пока вся энергия таких колебаний не перейдет во внутреннюю энергию тела. Аналогичной моделью таких растяжений и сжатий является пружина. Так как в данной задаче все возникающие волны являются продольными, то механическую систему можно рассмотреть в виде одномерной задачи. В механической системе массы представляют частицы материала, упругие взаимодействия со стенкой, а так же внутри тела моделируются пружинами. Энергия запасенная в пружинах, а так же энергия движения частиц относительно общего центра масс является аналогом внутренней энергии.

В работе исследуется влияние значений масс и жесткостей пружин на конечное поведение системы,
определяется качественный характер поведения модели и оптимальные параметры системы:
минимальная и максимальная запасенная телом энергия, возможные состояния, при которых вся энергия переходит во внутреннюю, либо внутренняя энергия равна нулю после взаимодействия.


\addtocontents{toc}{\protect\setcounter{tocdepth}{2}}